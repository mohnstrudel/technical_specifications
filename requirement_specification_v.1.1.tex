\documentclass[12pt]{article} % Äîêóìåíò ïðèíàäëåæèò êëàññó article, à òàêæå áóäåò ïå÷àòàòüñÿ â 12 ïóíêòîâ.
\usepackage{ucs}
\usepackage[T1,T2A]{fontenc}
\usepackage[utf8x]{inputenc} % Âêëþ÷àåì ïîääåðæêó UTF8
\usepackage[russian]{babel} % Ïàêåò ïîääåðæêè ðóññêîãî ÿçûêà
\usepackage{titling} % Allows custom title configuration

%For image using
\usepackage{graphicx}

%For referencing within enumeration lists
\usepackage{enumitem}

%Packages for word-like comment style
\usepackage{todonotes}

%For a nicer reference
\usepackage{fancyref}

%For some math formulas if needed
\usepackage{mathtools}

\title{Макет технического задания} % Заглавие документа
\date{\today} % Дата создания

\begin{document}
\maketitle

\section{Вводная}
Сайт представляет собой электронную версию журнала "Тель-Авив - Москва". Пользователь может просматривать предоставленную информацию - читать новости, записи в блогах, просматривать видео. Также пользователь может принимать активное участие в доступных для него областях сайта - писать комментарии, записываться на события.

\subsection{Обозначения}

\begin{itemize}
\item Блок - некий визуальный элемент, выделяющийся либо графически (в виде рамок, очертаний), либо по смыслу (совокупность похожих элементов)
\item Фронтэнд - для пользователя видимая оболочка веб-страницы
\item Бэкэнд - невидимые для пользователя математические алгоритмы
\item CMS - все работы происходят на основе системы управления содержанием - CMS 1C Bitrix (1С Битрикс)
\item Модуль - является описанием общего функционала, который не может быть классифицирован как привязанный к определенной странице. Он может встречаться на любой странице в любом месте
\item Хэдер - верхняя часть сайта, обладающая определенной структурой, которая видна сквозняком на всех или почти всех страницах сайта. Также используется обозначение "шапка".
\item Футер - нижняя часть сайта. Функционал аналогичен хэдеру. Также используется обозначение "подвал".
\end{itemize}

\subsection{Общая структура сайта}

На сайте представлены 

\section{Фронтэнд}
\subsection{Модули}


\begin{enumerate}
\item Статья - главный и самый важный элемент содержания; является страницей, которая может содержать как текст, изображения так и видео. Статья имеет перечень свойств (например категории и рубрики), которые описаны ниже:
    \begin{enumerate}
    \item{Категория} - категория сайта, к которой относится статья. Таких категорий несколько: новости, блоги, видеохроника и события. Не у всех пользователей есть права писать в каждую категорию. Например новые записи в новости, видеохронику и в события может добавлять только администратор, в то время как в в блоги новые записи может добавлять более широкий круг лиц. У каждой категории может быть неограниченное количество \label{subcategory} подкатегорий. Для новостей это может быть "мировые новости" или "новости раввината".
    \item{Рубрика} - представляет собой де факто также категорию, которая не выводится на сайте как таковая, т.е. нет ссылки в меню с названием \quotedblbase{Рубрики}, а сразу выводятся подрубрики. Таких 13 - календарь, колумнисты, мнение раввина, точка зрения, бизнес, герои, личности, быть евреем, кино, книги, благотворительность, чтение, красота, роскошь, кошерная мода, мировая мода, мода Израиля, дети, еда.
        \begin{figure}[ht!]
        \centering
        \includegraphics[width=90mm]{ss_10-09-2014.jpg}
        \caption{Меню в шапке \label{overflow}}
        \end{figure}
    \item{Короткое описание} - поле, содержащие не полный текст статьи. Применение - см. пункт \ref{announce}.
    \item{Главный текст} - поле, содержащее главный текст статьи. Открывается отдельной страницой и показывает все содержимое.
    \item{Баннеры} - помимо картинок/видео в теле статьи пользователь может загрузить отдельные баннеры, которые исполняют роль визуального заголовка статьи (для каждого баннера пользователь может подгрузить описание, автор сохраняется автоматически)
    \item{Тэги} - для каждой статьи можно сохранять набор тегов
    \item{Рейтинг} - каждая статья может оцениваться пользователями по десятибальной шкале
    \item{Дата, место и участники} - если создается статья в категории "события", то для нее нужно указать дату события и место проведения события. Также можно добавить участников встречи (т.е. пригласить других пользователей). 
    \end{enumerate}
\item \label{announce}{Анонс} (= превью) - анонс представляет собой часть полноценной статьи, которую может написать администратор, модератор или любой другой пользователь, обладающий соответствующими правами.

\end{enumerate}

\subsection{Страницы}
\subsubsection{Список статей}
Данная страница представляет собой блок информации, состоящей из предпросмотра баннера , анонса текста (см. \ref{announce}) и системной информации о статье
\begin{enumerate}
    \item Предпросмотр баннера является уменьшенным вариантом баннера, также на данном элементе расположена полоса \label{whiteline}, подгружающая в зависимости от принадлежности статьи различное содержание
    \begin{enumerate}
        \item Новости - название субкатегории
        \item Блоги - автор блога \todo[inline]{Или может быть также субкатегории блогов?}
        \item Видеохроника - название субкатегории
        \item События - место проведения события
        \item Журнал - главная тема архивного журнала
    \end{enumerate}
    \item Заголовок (текстовый) \& анонс текста
    \item Системная информация о статье - показываем дату создания статьи (формат: сегодня/вчера/конкретная дата + время), комментарии и рейтинг
\end{enumerate}

\subsubsection{Читайте также}
Данная страница (которая обычно является частью других страниц) подгружает X статей (X управляется в админке). В элементе полоски показываем принадлежность статьи к категории или рубрики (например это новость или видеоролик), так как в статьях разных разделов могут быть одинаковые теги. \todo[inline]{Так ли это?}

\subsubsection{Главная}
На главной странице отображаются несколько составляющих. Формат главной - две колонки (главная и правый сайдбар).

\begin{enumerate}
    \item Горячая статья - отображаем слайдером список из X горячих статей (X управляется в админке), если кол-во статей, которые редактор пометил как горячие превышает значение X, то отображаем X самых актуальных. Блок с горячей статьей состоит из большой картинки и слоя (не полностью закрывающего картинку) с текстовым заголовком, превью текста статьи и (см. \ref{whiteline}) элемента полосы. Для горячих статей в этом элементе выводим значение рубрики или категории.
    \item Место для рекламного баннера
    \item Самое популярное - отображаем статьи любых рубрик и категорий с наивысшим рейтингом.
    \item Цитата дня - отдельно редактируемый сниппет, который подргужает цитату случайным образом из списка цитат (список редактируется в админке).
    \item Блоги - подгружаем \todo[inline]{Действительно самые последние? Или другие критерии?} самые последние статьи из категории "блоги".
    \item Наши герои - Подгружаем последние актуальные статьи, относящиеся к рубрике "герои".
    \item Видеохроника - аналогично блогам и героям.
    \item Сайдбар - состоит из списка новостей, места для рекламного баннера, календаря и списка событий.
        \begin{enumerate}
            \item Список новостей состоит из последних записей в статьях категории "новости". При этом верстка самой актуальной статьи отличается от её предшественниц (см. изображение \ref{fig:main_right_sidebar.jpg})
            \begin{figure}[ht!]
            \centering
            \includegraphics[width=90mm]{main_right_sidebar.jpg}
            \caption{Правый сайдбар \label{fig:main_right_sidebar.jpg}}
            \end{figure}
        \end{enumerate}
    \item События - элемент, позволяющий выбрать дату и подгрузить в блоке со списком событий перечень событий \todo[inline]{Или же открываем в новой странице?} выбранного дня.
    \item Список событий конкретного дня (по умолчанию показываем самые актуальные, начиная с даты "/today"). Список делится на события журнала, события участников и приглашения.
        \begin{enumerate}
            \item События журнала являются событиями (статьями), созданными администратором или соответствующе помеченным пользователем. Флажок событий - желтый
            \item События участников являются событиями (статьями), созданными модераторами/редакторами/авторами. Флажок событий - розовый.
            \item Если пользователь приглашен на любое событие, то оно помечается зеленым флажком.
        \end{enumerate}
    \item В конце сайдбара находится гиперссылка, ведущая на страницу со всеми событиями месяца. Данная страница представляет собой перечень анонсов событий.
\end{enumerate}

\subsubsection{Футер сайта}
В нижней части сайта кроме текстового содержания 4 столбца. Наполнение уточняется, на данный момент это - первый столбец: разделы сайта (категория), второй и третий столбцы: специальные текстовые страницы, выводимые только в футере. Четвертый столбец - иконки соц.сетей.

\subsubsection{Личный кабинет}
Описание данного подраздела включает в себя регистрацию, страницу логина и сам личный кабинет.

\paragraph{Регистрация}
Пользователь может зарегистрировать себя как через соц. сеть Facebook, так и полностью заполнив необходимые поля. 
При регистрации через Facebook за пользователя заполняются уже необходимые поля из данных его учетной записи в Facebook и создается новая запись в базе данных. При манульной регистрации пользователь должен заполнить все \todo[inline]{Нужен перечень полей и сама страница регистрации} необходимые поля.

\paragraph{Вход на сайт}
Пользователь вводит логин \& пароль после чего система перенаправляет его в личный кабинет.

\paragraph{Личный кабинет (ЛК)}
В ЛК пользователь видит что-то \todo[inline]{Необходим список того, что видит пользователь в ЛК - какие страницы должны быть в ЛК}

%Референс - \ref{sec:testlabel} \pageref{sec:testlabel}

\subsubsection{Страница статьи}
На странице конкретной статьи представлена полноценная информация по статье. Данная информация состоит из:

\begin{enumerate}
    \item Хлебных крошек в виде визуальных элементов. 
    \item Заголовок (текст), иконка и кол-во комментариев, иконка и рейтинг пользователей.
    \item Слайдер с баннер(ы) статьи (визуальный заголовок статьи), на них присутствует также элемент полупрозрачной полосы (см. \ref{whiteline} ). На данном элементе отображаем автора и примечание к картинке. Если у статьи много баннеров, то слайдер показывает элементы для перехода между баннерами.
    \item Боди статьи выводит главное содержание статьи (текст, видео, дополнительные изображения)
    \item Подвал статьи - содержит автора и дату публикации
    \item Социальное - содержит модуль pluso.ru, позволяющий рекомендовать статью во всех доступных соц. сетях. Также присутствует собственный метод оценки (каждый пользователь может голосовать только один раз, незарегистрированные не могут голосовать вообще)
\end{enumerate}


\section{Бэкэнд}

\subsection{Популярные материалы}
Рейтинг статей рассчитывается из голосований пользователей. Строится обычная средневзвешенная оценка.
\begin{equation}
    \frac{1}{n} \sum\limits_{n=1}^{N} rating_n     
\end{equation}


\end{document}
